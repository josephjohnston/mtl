

\section{Functions}

kernel, vertex, and fragment attributes, short for [[kernel]], [[vertex]], [[fragment]]
also have [[visible]], [[stitchable]], [[intersection(..)]], [[object]], [[mesh]]
these attributes at the start of a function make it accessible to the Metal API, making it a qualified function (for us only qualified functions are kernel and visible and stitchable)
kernel, vertex, and fragement attributes can call visible functions.
refer to qualified function f in namespace np as np::f.

functions declared with kernel attribute must return void
can specify [[max_threads_per_threadgroup(N)]], fails if N is greater than device.maxThreadsPerThreadgroup
[[visible]] functions are visible from the Metal API, ie you can get a MTLFunction from it can get a pointer to a visible function, which can be used with the visible_function_table. other functions can directly call a visible function.
[[visible]] float my_visible(device int *data, int data_offset) {}
[[stitchable]] functions can be used in Function Stitching API. stitchable implies visible. compiler will generate some metadata, increasing code size, only use if necessary.
[[stitchable]] float my_stitchable(device float *data, texture2d<float> tex) {}

use [[host_name(name)]] to use name in the API

can use templates with qualified function declarations
template<typename T>
kernel void bar(device T *x) {}
template kernel void bar(device int *x);
template [[host_name("bar_int")]] kernel void bar(device int *x);
template [[host_name("bar_float")]] kernel void bar(device float *x);

Function Arguments and Variables
most inputs pass as arguments, except for vars in constant address space and sample (only?) in program address space.
argument include
1. device buffer (any pointer or ref to any data type in device space)
2. constant buffer (ditto for constant space)
3. texture object
4. texture_buffer
<!-- 5. sampler -->
6. pointer to threadgroup address space (buffer shared between threads in threadgroup)
<!-- 7. imageblock -->
8. argument buffer
9. visible function table
<!-- 10. intersection function table -->
<!-- 11. acceleration structure -->
12. structure with elements buffers, textures, texture buffers. 

buffers in device or constant space can't alias for same function.
can't use size_t (or anyting containing size_t) for arguments
attributes help specify arguments regarding resource location and built-in variables (for communication with fixed-function stages)

optional attribute for any argument specifying location 
device and constant buffers: [[buffer(index)]]
textures (including texture buffers): [[texture(index)]]
sampler: [[sampler(index)]]
threadgroup buffers: [[threadgroup(index)]]
index is unsigned

kernel void add_vectors(
    const device float4 *inA [[buffer(0)]],
    const device float4 *inB [[buffer(1)]],
    device float4 *out [[buffer(2)]],
    uint id [[thread_position_in_grid]]
) {
    out[id] = inA[id] + inB[id]
}

[[kernel]] void
my_kernel(
    device float4 *p [[buffer(0)]],
    texture2d<float> img [[texture(0)]],
    sampler sam [[sampler(1)]]
) {}

kernel void
my_kernel(
    texture2d<half> src,
    texture2d<half, access::write> dst,
    sampler s,
    device myUserInfo *u
) {}

attributes buffer, texture, sampler, threadgroup represent groups of resources
indices must be unique within each group

struct VertexOutput {
    float4 position [[position]];
    float4 color;
    float2 texcoord;
};
struct VertexInput {
    float4 position;
    float3 normal;
    float2 texcoord;
};
constexpr const uint MAX_LIGHTS = 4;
struct LightDesc {
    uint num_lights;
    float4 light_position[MAX_LIGHTS];
    float4 light_color[MAX_LIGHTS];
    float4 light_attenuation_factors[MAX_LIGHTS];
}
vertex void
render_vertex(
    const device VertexInput* v_in [[buffer(0)]],
    constant float4x4& mvp_matrix [[buffer(1)]],
    constant LightDesc& light_desc [[buffer(2)]],
    device VertexOutput* xform_output [[buffer(3)]],
    uint v_id [[vertex_id]],
) {
    VertexOutput v_out;
    v_out.position = v_in[v_id].position * mv_matrix;
    v_out.color = do_lighting(v_in[v_id].position,v_in[v_id].normal, light_desc);
    v_out.texcoord = v_in[v_id].texcoord;
    xform_output[v_id] = v_out;
}

inputs can be had by intedxing into buffers with ids, or by using [[stage_in]]


