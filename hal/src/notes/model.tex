

launch a kernel and blocks get scheduled to processors as they become available.
once a block is active on a processor, meaning resources are allocated like registers and threads, it will remain resident until completion. processor resources are distributed among it's currently active blocks
in a block, all threads are made resident until they retire. 
groups are scheduled (like blocks) as resources (cores) become available.
barriers can synchronize all threads in a block, or all threads in a group.
? are groups scheduled in order? why not?

not bad to imagine as if only one processor.
and for most cross communication, aiming for a single resident block at a time, without enough warps to hide all latency.

dimensions x,y,z from least to most significant.
in hardware a block is a 1d array of groups, or 1d array of threads paritioned into groups.



threadgroup width may be smaller than execution width, in which case groups have multiple rows.

maximize active warps (minize registers per warp) (also TLP)
latency * throughput
or maximize ILP