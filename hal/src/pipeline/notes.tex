pipelines

DYNAMIC LIBRARIES

one of the two library types.
have an existing library, either in metallib or Library, and pass either one to createDynamicLibrary of device. input is a library, output is a dynamic library. these are the two library types. 
you can serialize the dynamic library to a metallib file.
to reload a serialized dynamic lib, use newDynamicLibraryWithURL
when you reload an archive, it takes memory, so this is why you want to split chaches up. 

I still don't understand the purpose and usage of installName.

I think here stack depth is important, which also is a property of the pipeline descriptor.
any chain of dynamic function calls must consider stack depth.

we can get dynamic libraries to work when creating the executable from a source string, but we don't know for the case of a metallib file (we don't know how to compile the file with externs that it doesn't compile but waits for linkage at runtime).

wait for updates on questions, adjust as necessary, add support for dynamics later if deemed useful.


LINKED FUNCTIONS

we don't need external static linkage since we are authors of all shaders, and should two be created at different times and statically linked, we can combine them as sources.
we don't use the other pars of linked fnctions since they are visible functions, i think equivalent to function pointers, and those are not performant. 


SOURCE CODE

here we need to start writing source coee. we need to figure out what the grammar and semantics are and what valid code looks like. 

when we create libraries, 
we'll archive pipelines, which hold only single funcitons. and i think we'll compile these functions individually. or we could compile libraries holding similar functions, this way we avoid duplicaiton of compilation. then we'll take the library right after and split/duplicate into pipeline functions, archive those, and dispose of the lbirary. henceforth there's duplciation. we could always regenerate the lbirary and rebuild pipelines.
from the outside it should look like the ability to write a library of functions, select out qualified (we only have kernel) functions, compile and cache them to a pipeline, then later load the pipeline. of course you can save libraries and functions, and rebuild later. but this rebuilding process, which is both compilation from source, and pipeline creation, is a single expensive process. 


Stitched functions:

graph. nodes are functions. connect inputs and outputs. 

Binding:
access: read, write, readwrite
index: usize
argument: bool
used: bool
name: string
type: buffer, texture, threadgroup memory, 

Buffer binding:
alignment: usize,
data size: usize,
data type: dataType,
pointer type: pointerType

Pointer type:
alignment: usize,
data size: usize,
element type: dataType,
access: read, write, readwrite,
element is arg buffer: bool,
element array type: arrayType,
element struct type: structType,

Array type:
array length: usize,
element type: dataType,
stride: usize,

so now we compile all by file, build files, archive or not, just an api that programs and delivers pipelines. they should be distinct. note only relevant for compute.







